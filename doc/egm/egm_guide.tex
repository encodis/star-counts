%%----------------------------------------------------------------------------|
%% EGM USERS GUIDE                                                            |
%%                                                                            |
%% P. J. C. Hodder                                                            |
%%----------------------------------------------------------------------------|

\documentstyle[11pt,twoside,psfig]{article}

%%----------------------------------------------------------------------------|
%% EGM USERS GUIDE - MACRO DEFINITIONS                                        |
%%                                                                            |
%% P. J. C. Hodder                                                            |
%%----------------------------------------------------------------------------|

\textheight 220mm
\textwidth 160mm
\oddsidemargin 6.6mm
\evensidemargin 0mm
\topmargin -6.4mm

\pagenumbering{arabic}
\renewcommand{\thepage}{\fbox{\em \arabic{page}}}
\pagestyle{myheadings}
\markboth{\fbox{EGM: A User's Guide}}{\fbox{EGM: A User's Guide}}

\newcommand{\egm}{{\sc egm}}
\newcommand{\egmsm}{{\sc egm.sm}}
\newcommand{\egmread}{{\sc egm\_read}}
\newcommand{\egmplot}{{\sc egm\_plot}}
\newcommand{\egminfo}{{\sc egm\_info}}
\newcommand{\egmdiff}{{\sc egm\_diff}}
\newcommand{\egmint}{{\sc egm\_int}}
\newcommand{\egmcol}{{\sc egm\_col}}
\newcommand{\egmprint}{{\sc egm\_print}}
\newcommand{\egmver}{egm--1.0.tar.Z}
\newcommand{\sm}{{\sc sm}}
\newcommand{\lf}{LF}
\newcommand{\cmd}{CMD}
\newcommand{\fms}{FMS}
\newcommand{\mlr}{MLR}
\newcommand{\nin}{\noindent}
\newcommand{\sci}[2]{$#1 \times 10^{#2}$}

\newenvironment{clo}[1]{
\noindent{\tt {#1}} \\
\indent\begin{minipage}[t]{6in}}
{\bigskip\end{minipage}}

\newenvironment{key}[3]{
\noindent{\tt {#1} [{#2}] {#3}}\\
\indent\begin{minipage}[t]{6in}}
{\bigskip\end{minipage}}

\renewcommand{\floatpagefraction}{0.1}
\renewcommand{\bottomfraction}{0.5}

\pssilent


\begin{document}  

%%----------------------------------------------------------------------------|
%% TITLE PAGE                                                                 |
%%----------------------------------------------------------------------------|

\thispagestyle{empty}
\noindent
\framebox[16cm]{\parbox{16cm}{
\begin{center}
\huge \bf 
A Users Guide to EGM -- 

\rule{0mm}{10mm}

A Galaxy Modelling Program

\rule{0mm}{10mm}

Version 1.7

\end{center}}}

\rule{0mm}{10mm}

\noindent
\framebox[16cm]{\parbox{16cm}{
{\begin{center}
\large \bf
Contents
\end{center}}
\large
\begin{tabbing}
\rule{2cm}{0cm} \= \rule{1cm}{0cm} \= \rule{1cm}{0cm} \= \rule{9cm}{0cm} \= \kill
\> 1  \> Introduction \> \> ~1 \\ \\
\> 2  \> A Brief Review of EGM \> \> ~1\\
\> \> 2.1 \> The Model \> ~1 \\
\> \> 2.2 \> Parameter Files \> ~1\\
\> \> 2.3 \> Running EGM \> ~2 \\
\> \> 2.4 \> Command Line Options \> ~3 \\
\> \> 2.5 \> Error Codes \& Runtime Statistics \> ~5 \\ \\
\> 3  \> Parameter Description and Formats \> \> ~6 \\
\> \> 3.1 \> Adjustable Parameters \> ~6 \\
\> \> 3.2 \> File Formats \> 10 \\
\> \> 3.3 \> Output Format \> 10 \\ \\
\> 4  \> Model Description \> \> 22 \\
\> \> 4.1 \> Geometrical Considerations \> 22 \\
\> \> 4.2 \> Component Models \> 24 \\
\> \> 4.3 \> Distribution Functions \> 26 \\
\> \> 4.4 \> Program Function Details \> 27 \\ \\
\end{tabbing}
}}

%%----------------------------------------------------------------------------|
%% INTRODUCTION                                                               |
%%----------------------------------------------------------------------------|

\newpage
\setcounter{page}{1}
\section{Introduction}

This guide is divided into 4 main sections.  This is the first section (not
surprisingly!). Section 2 is a brief overall review of the model and how
to run \egm. Section 3 describes the model parameters in more detail, and
Section 4 contains background information and references for some of the
scientific background. 

Bug reports and suggestions are welcome. The author of \egm\ (Philip Hodder)
can be contacted at the Department of Geophysics and Astronomy, University of
British Columbia, \#129-2219 Main Mall, Vancouver, B.C., Canada V6T 1Z4, or
via E--mail at {\tt hodder@geop.ubc.ca}.

%%----------------------------------------------------------------------------|
%% A BRIEF REVIEW OF EGM                                                      |
%%----------------------------------------------------------------------------|

\section{A Brief Review of EGM}
 
%%----------------------------------------------------------------------------|
\subsection{The Model}

The \egm\ program computes number counts and colour distributions for a
model of an (external) spiral galaxy containing at two or three components.
Using user supplied luminosity functions and colour-magnitude diagrams \egm\ 
will integrate along a line of sight to a particular field in the galaxy,
down to the desired limiting magnitude.  Almost every parameter 
of the model can be altered by the user by specifying it in a parameter file
or on the command line -- reasonable defaults are used for all other
parameters (see \S 3).

Currently the forms of the density distributions of the components are
fixed. The disk component is an exponential disk (varying in height
and length) whose scale lengths can be specified. The form of the spheroidal 
component is a de Vaulcouleurs ${\rm r}^\frac{1}{4}$ spheroid. The
optional third component is also an exponential disk whose parameters may
be varied independently of the first, thin, disk.

%%----------------------------------------------------------------------------|
\subsection{Parameter Files}

\egm\ is run entirely from the command line -- it has no direct interaction
with the user in the form of prompts, queries and so on, although it will
print warnings before it quits when it cannot do something. A
{\em parameter file} contains all the necessary information to successfully
produce a model -- it is therefore vital that the user fully understand
all the necessary elements of this file. This isn't as daunting as it 
sounds -- whilst there are 56 parameters that {\em can} be specified, you can
usually get away with using 7 or 8. The others may be changed if needed.
Most of the commonly changed parameters can also be specified on the
command line. Values in a parameter file will override any command line 
arguments if it occurs after them, and vice versa.

The parameter file contains specifications for the field you wish to
integrate along (in arc minutes along the major and minor axes), the field 
size (in square degrees), the magnitude limits, the files containing the
luminosity functions and colour--magnitude diagrams, and so on. 
Note that unlike the original Bahcall \& Soneira (B\&S) code, \egm\ will 
work with luminosity functions (\lf s), colour magnitude diagrams (\cmd s)
not just those in $V$ and $B-V$. This is because the user supplies this 
data -- it is not calculated from a hard--coded analytical formula during 
execution of the program, or from a function call.

Section 3 contains a list of all the parameters and their defaults. 
It also lists their {\em keywords} -- i.e. the words in the parameter file 
that \egm\ recognizes. For example, to specify a field size of 0.6 square 
degrees, you could put this line in your parameter file:
\begin{verbatim}
OMEGA = 0.6
\end{verbatim}
All keywords must be in uppercase, and the keyword and the actual value
must be separated by an equals sign. There can be any number of spaces
around the {\tt =}. Any line beginning with a \# is treated as a comment and is
ignored.

I recommend that for the sake of consistency all parameter files end
in ``.pm'' -- \egm\ comes with ``default.pm'' as an example. The \cmd\ files
end in ``.(mag)(col)'' (e.g. m5.vbv is a \cmd\ for M5 in V and B--V). Likewise
the \lf s end in ``.(mag)lf'' (e.g. analytic.vlf).
Although \egm\ does not care what these files are called, it 
{\em does} care how the data inside it is arranged (see \S3 for a description
of the various file formats).

\egm\ will look in several places for these files. Firstly it
will look in the current directory. Then it will look for them in the
directory specified by the {\tt LIBRARY} keyword in the parameter file.
Next it will check the command line for the {\tt -L} option.
Lastly it will look in a standard 
library. This is a directory containing a selection of \lf s and \cmd s whose 
pathname is compiled into \egm\ during installation.
You can of course add files to this library as you need them.
\egm\ will quit (and complain) if it can't find the required files somewhere.
An example of a parameter file is shown below:
\begin{verbatim}
nukta-1> cat default.pm
# Model Parameters #
X0        = 50.0
Y0        = 90.0
OMEGA     = 1.0
DGC_FILE  = dskm67.vbv
SGC_FILE  = m13.vbv
DLF_FILE  = analytic.vlf
SLF_FILE  = analytic.vlf
\end{verbatim}

%%----------------------------------------------------------------------------|
\subsection{Running EGM}

So, once you have your parameter file set up (you can use your favourite
text editor to do this -- it is just a text file) you can run \egm\ by 
typing the following at the system prompt:
\begin{verbatim}
nukta-2> egm my_params.pm
\end{verbatim}
The model will then be printed out to the terminal/xterm/whatever. Since
the output can be 600 to 700 lines long this is not terribly
useful. I recommend you redirect the output to a file of your choice:
\begin{verbatim}
nukta-3> egm my_params.pm > my_results.dat
\end{verbatim}
Alternatively,
\begin{verbatim}
nukta-3> egm my_params.pm -o my_results.dat
\end{verbatim}
\egm\ (note that the program name is ``egm'', in lowercase) is usually quite
quick to run, but you may want to put it in the background anyway:
\begin{verbatim}
nukta-4> egm my_params.pm > my_results.dat &
\end{verbatim}
\egm\ outputs the model results in one big file (though this may change in 
future versions).
There is a short header section at the front of the file. This is
essentially a copy of all relevant parameters. The number counts are printed  
out next followed by the colour distribution. Additional blocks of data may 
then be printed (depending on various options set on the command line) and
each block is prefaced with an identifier. Section 3.3 describes the output
in more detail.

You can now split this file up into the separate pieces and display it using
whatever plotting package turns you on! \egm\ does NO plotting itself -- a
design goal was to make it as machine independent as possible.
A macro package for SuperMongo (\sm) is included in the source code
archive. This
takes the whole file output by \egm\ and plots the number counts and
colour distributions automatically. See Appendix B for more information.
 
%%----------------------------------------------------------------------------|
\subsection{Command Line Options}

There are several command line options that can be given to \egm. These
options override those parameters given in the parameter file (which, in turn,
override the default values supplied by the program).
These options can be useful if you want to run the same model
for a different field and don't want to create another parameter file.
The options are:
\medskip

\begin{clo}{-a absorption\_mode}
The absorption mode to use in the model. Allowable choices are 'none', 'cosec'
and 'Sandage' for no absorption, the cosecant law and the Sandage model
respectively. Only the first letter of the word is important, and it may 
be in upper or lower case.
\end{clo}

\begin{clo}{-A absorption\_value}
A fixed amount of absorption to include in the model (in magnitudes).
\end{clo}

\begin{clo}{-bm magnitude\_bin}
Size of magnitude bin for scaling differential number counts on output.
\end{clo}

\begin{clo}{-cb magnitude}
The brightest magnitude of stars to be included in the colour distribution.
\end{clo}

\begin{clo}{-ce colour\_error}
The colour error in magnitudes. The default is 0.1.
\end{clo}

\begin{clo}{-cf magnitude}
The faintest magnitude of stars to be included in the colour distribution.
\end{clo}

\begin{clo}{-cl colour}
The minimum colour for the colour distribution.
\end{clo}

\begin{clo}{-cu colour}
The maximum colour for the colour distribution.
\end{clo}

\begin{clo}{-dm magnitude\_interval}
The apparent magnitude interval for the number counts.
\end{clo}

\begin{clo}{-dM magnitude\_interval}
The absolute magnitude interval for number counts and colour distribution.
\end{clo}

\begin{clo}{-dn disk[:thick\_disk]:spheroid}
The density normalization radii for the galaxy model components, in parsecs
\end{clo}

\begin{clo}{-D distance}
The distance to the galaxy (in parsecs).
\end{clo}

\begin{clo}{-es eccentricity}
The eccentricity of the spheroid component.
\end{clo}

\begin{clo}{-f field\_size}
The size of the field in square degrees.
\end{clo}

\begin{clo}{-F}
This tells \egm\ to integrate all components on the far side of the galaxy 
(i.e. on the other side of the disk) as well as the near side. The default is 
to only use the side closest to the observer, but this may change in future 
versions. (See \S4 for further information).
\end{clo}

\begin{clo}{-FD}
\egm\ will integrate the near and far sides of the disk component, and the only
the near sides for the other components.
\end{clo}

\begin{clo}{-FS}
\egm\ will only integrate the far side of the spheroid component.
\end{clo}

\begin{clo}{-FT}
\egm\ will only integrate the far side of the thick disk component.
\end{clo}

\begin{clo}{-h}
This option prints a list of command line options to the standard output 
stream.
\end{clo}

\begin{clo}{-H}
This option prints a list of parameter keywords to the standard output stream.
\end{clo}

\begin{clo}{-i inclination}
The inclination of the disk of the galaxy (in degrees).
\end{clo}

\begin{clo}{-L path}
Specifies an alternate directory for the \lf\ and \cmd\ files.  The
default is  ``/home/hodder/lib/egm/''  (note the ending ``/'').  The
keyword {\tt LIBRARY}  may also be  used in the parameter file for  this
purpose.
\end{clo}

\begin{clo}{-mb magnitude}
The brightest apparent magnitude to be included in the model.
\end{clo}

\begin{clo}{-mf magnitude}
The faintest apparent magnitude to be included in the model.
\end{clo}

\begin{clo}{-Md magnitude}
The brightest absolute magnitude of the disk luminosity function that is
to be included.
\end{clo}

\begin{clo}{-Mf magnitude}
The faintest absolute magnitude of both the disk and spheroid luminosity
functions that is to be included in the model.
\end{clo}

\begin{clo}{-Ms magnitude}
The brightest absolute magnitude of the spheroid luminosity function that is
to be included.
\end{clo}

\begin{clo}{-MS}
This tells \egm\ to include a main-sequence colour-magnitude diagram in the
calculations. The actual file name to use for this \cmd\ must be specified
in the parameter file using the {\tt MSC\_FILE} keyword. (See \S4 for further 
information).
\end{clo}

\begin{clo}{-o filename}
Sends the output of \egm\ to {\tt filename}, rather than standard output.
\end{clo}

\begin{clo}{-P "parameter file entry"}
This is an alternate way to specify any parameter that has a parameter file
keyword. The string after the {\tt -P} should be of the same format as an 
entry in the parameter file. For example, there is no explicit option to
set the disk \lf\ but one could include {\tt -P "DLF\_FILE = foo.vlf"} on
the command line to set it without editting the parameter file.
\end{clo}

\begin{clo}{-re radius}
The effective (half--light) radius of the spheroid (in parsecs).
\end{clo}

\begin{clo}{-sh disk\_scale\_height}
The scale height of the disk, in parsecs.
\end{clo}

\begin{clo}{-sl disk\_scale\_length}
The scale length of the disk, in parsecs.
\end{clo}

\begin{clo}{-T}
This tells \egm\ to include the thick disc in the calculations. The default
is to not include this component. The parameters of this thick disk must be
set in the parameter file if anything other than the defaults values are to
be used (see \S3).
\end{clo}

\begin{clo}{-V}
This turns on ``verbose mode''. \egm\ will print out extra data 
concerning the distribution functions used in the model. This is
mainly used for debugging purposes. See \S3.4 for more information.
\end{clo}

\begin{clo}{-x x\_position}
The position along the major axis in arc minutes. The relevant geometry is
described in section 4.1.
\end{clo}

\begin{clo}{-y y\_position}
The position along the minor axis in arc minutes.
\end{clo}

%%----------------------------------------------------------------------------|
\subsection{Error Codes \& Runtime Statistics}

This section will only be useful for a few people, but it should be written
down somewhere! If \egm\ runs to completion the normal exit status is 0.
If at any point during execution \egm\ encounters an error it will stop
with an error message and the appropriate exit status.

\begin{description}
\item[0] Indicates successful completion of program. No error message.
\item[1] A usage error. Usually this means that no command line options were
specified.
\item[2] An invalid command line option was used. See section 2.4, Command
Line Options, and try again.
\item[3] An invalid parameter was found in the parameter file. See section 3.1,
Adjustable Parameters, for a complete list of the allowed keywords.
\item[4] The file specified in one of the parameter file keywords does not 
exist. This may also mean that no parameter file was given on the command
line, or that the library of standard files does not exist or is
incorrectly specified. See Appendix~A for installation and compilation
guidelines.
\item[5] A file could not be read properly. The file is most likely in the
wrong format. See section 3.2, File Formats.
\item[6] Memory allocation error. \egm\ could not allocate enough memory
for one of its internal tables. See your friendly, neighbourhood systems
manager.
\item[7] A math error. Currently this will only be generated by the spline
interpolation routine. Make sure all the input files are monotonically
increasing in the first column.
\end{description}

\egm\ runs fairly quickly on most computer systems. The following times have
been recorded on three different setups. The model used was given by the 
default set of parameters. The {\tt -O} flag was used during compilation to
produce more optimised code.

\begin{tabbing}
\rule{1cm}{0cm} \= \rule{10cm}{0cm} \= \kill 
\> IBM PC Clone (25 MHz, 386DX, 387 Co-Processor) : \> 39.0 seconds \\
\> Sparc 1 (SunOS) [iridium] :                      \> 15.0 seconds \\
\> Sparc 2 (SunOS) [wendigo] :                      \> ~8.0 seconds 
\end{tabbing}

%%----------------------------------------------------------------------------|
%% PARAMETER DESCRIPTION AND FORMATS                                          |
%%----------------------------------------------------------------------------|

\section{Parameter Description and Formats}
 
%%----------------------------------------------------------------------------|
\subsection{Adjustable Parameters}

There follows a list of all the parameters that can be adjusted, though 
it usually won't be necessary to change most of these.
Those parameters marked with an asterisk ($*$)  are
ones that  I think should  at least be thought about before you run \egm,
even if you don't change them. Those marked with a plus ($+$) shouldn't be
changed unless you're really sure you  know what you're doing. The  list
is in no particular order -- the parameter file also does not need to be
arranged  in any  special way.  The list is  given  by  keyword with the
default in parentheses (note that most of these defaults apply to M31).
\medskip

\begin{key}{A\_0}{100.0}{}
The scale height for absorption/extinction (in parsecs).
\end{key}

\begin{key}{A\_1}{0.15}{}
The absorption in magnitudes at the Galactic poles (i.e. $a_1(90^\circ)$) in 
the ``cosec'' absorption law. The default is for the $V$ filter.
\end{key}

\begin{key}{A\_2}{0.165}{}
The $a_2$ parameter for the Sandage absorption law (see \S 4.2). The default 
value is for the $V$ filter.
\end{key}

\begin{key}{A\_3}{1.192}{}
The $a_3$ parameter for the Sandage absorption law (see \S 4.2). The default 
value is for the $V$ filter.
\end{key}

\begin{key}{ABS}{0.0}{}
An arbitrary value (in magnitudes) to include in the model. This can simulate
absorption in our own Galaxy.
\end{key}

\begin{key}{AMODE}{Sandage}{}
This keyword selects the absorption mode to be used for the model. Possible 
values are ``none'', ``cosec'' and ``Sandage'' to use no absorption,
the cosecant law or the Sandage obscuration model. Like the {\tt -a} option
only the first letter of the word is important and it may be in upper or
lower case.
\end{key}

\begin{key}{COL\_BIN}{0.20}{(*)}
The colour bin size for output. The colour distribution will be scaled to 
number of stars per ({\tt COL\_BIN}) magnitude interval.
\end{key}

\begin{key}{COL\_ERR}{0.10}{(*)}
The FWHM of the Gaussian error function convolved with the 
calculated colour distribution to produce a predicted colour distribution.
\end{key}

\begin{key}{COL\_NAME}{B-V}{}
The name of the colour scale. It has no effect on the model calculation.
\end{key}

\begin{key}{CON\_FAC}{1.0E-6}{(+)}
A convergence criterion for ceasing integration. When the contribution to
the number counts of one element is less than {\tt CON\_FAC} times the total, 
the integration is stopped. (See section 4.4.)
\end{key}

\begin{key}{DEL\_ABS}{0.05}{}
The absolute magnitude interval used during calculation.
\end{key}

\begin{key}{DEL\_APP}{0.5}{}
The apparent magnitude interval used during calculation.
\end{key}

\begin{key}{DEL\_COL}{0.005}{(+)}
The colour interval for calculation before convolution.
\end{key}

\begin{key}{DGC\_FILE}{dskm67.vbv}{(*)}
The file containing the colour--magnitude information for the disk giants.
\end{key}

\begin{key}{DISTANCE}{725000.0}{}
The distance to the galaxy in parsecs.
\end{key}

\begin{key}{DLF\_FILE}{wielen.vlf}{(*)}
The file containing the \lf\ for the disk.
\end{key}

\begin{key}{DNR}{100:100 $|$ 10:10:10}{(*)}
The density normalization radii, in parsecs, for each of the components. For
a two component model the string refers to disk and spheroid; for the three
comppnent model it refers to disk, thick disk amd spheroid. These values are
used to calculate the actual density normalization.
\end{key}

\begin{key}{DR\_D}{25.0}{}
The distance increment for integration in the disk (in parsecs).
\end{key}

\begin{key}{DR\_T}{25.0}{}
The distance increment for integration in the thick disk (in parsecs).
\end{key}

\begin{key}{DR\_S}{1000.0}{}
The distance increment for integration in the spheroid (in parsecs).
\end{key}

\begin{key}{DSK\_ABS}{1.0}{}
An extra amount of absorption included when integrating through the disk (i.e.
it applies to calculations performed on the far side of the model only).
\end{key}

\begin{key}{E\_SPHER}{0.8}{}
The spheroid axis ratio.
\end{key}

\begin{key}{FAR\_DISK}{FALSE}{}
Integrate the far side of the disk in addition to the ``near'' side.
\end{key}

\begin{key}{FAR\_SIDE}{FALSE}{}
Integrate the far side of all the components in addition to the ``near'' side.
\end{key}

\begin{key}{FAR\_SPHD}{FALSE}{}
Integrate the far side of the spheroid in addition to the ``near'' side.
\end{key}

\begin{key}{FAR\_THIK}{FALSE}{}
Integrate the far side of the thick disk in addition to the ``near'' side.
\end{key}

\begin{key}{FMS\_FILE}{analytic.vfm}{(*)}
The file containing the fraction of stars on the main sequence as a function
of absolute magnitude. This is only needed when the main sequence is being
included.
\end{key}

\begin{key}{GSH\_D}{250.0}{}
The scale height of (the giants in) the disk (in parsecs).
\end{key}

\begin{key}{GSH\_T}{1300.0}{}
The scale height of (the giants in) the thick disk (in parsecs).
\end{key}

\begin{key}{INC}{77.5}{}
The inclination of the disk in degrees. This is defined so that an 
inclination of $90^{\circ}$ is edge on to the observer.
\end{key}

\begin{key}{LIBRARY}{/home/hodder/lib/egm}{}
The standard library of \lf, \cmd\ and \fms\ files, compiled into \egm\ during 
installation.
\end{key}

\begin{key}{M\_BRI}{-6.0}{}
The absolute magnitude of the bright end cutoff for all the \lf s.
\end{key}

\begin{key}{M\_DIM}{16.5}{}
The absolute magnitude of the faint end cutoff of all \lf s.
\end{key}

\begin{key}{MA\_BRT}{16.0}{}
The brightest apparent magnitude used in the model.
\end{key}

\begin{key}{MA\_DIM}{26.0}{}
The limiting apparent magnitude of the model. The integration will not be 
continued for stars fainter than this limit.
\end{key}

\begin{key}{MAG\_BIN}{0.0}{}
The bin size to use for scaling the counts. A value of 
0.0 indicates that no scaling occurs.
\end{key}

\begin{key}{MAG\_NAME}{V}{}
The name of magnitude scale. It has no effect on the model calculation, but 
has been included for completeness.
\end{key}

\begin{key}{MAX\_COL}{3.0}{(*)}
The maximum colour of the colour distribution. The default value of 3.0 
refers to the default $B-V$ colour.
\end{key}

\begin{key}{MBC}{16.0}{}
The brightest apparent magnitude of stars to be included in the colour 
distribution.
\end{key}

\begin{key}{MFC}{26.0}{}
The faintest apparent magnitude of stars to be included in the colour 
distribution.
\end{key}

\begin{key}{MIN\_COL}{0.0}{(*)}
The minimum colour of the colour distribution. The default value of 0.0 
refers to the default $B-V$ colour.
\end{key}

\begin{key}{MS\_FILE}{ms.vbv}{(*)}
The file containing the colour--magnitude information for the (optional)
main sequence.
\end{key}

\begin{key}{OMEGA}{0.001}{(*)}
The size of the field (in square degrees).
\end{key}

\begin{key}{PSL\_D}{5500.0}{}
The scale length of the disk (in parsecs).
\end{key}

\begin{key}{PSL\_T}{5500.0}{}
The scale length of the thick disk (in parsecs).
\end{key}

\begin{key}{R\_CO}{100.0}{}
The scale radius used in the Caldwell \& Ostriker model, if it is enabled.
\end{key}

\begin{key}{R\_EK}{2670.0}{}
The de Vaucouleurs radius of the spheroid (in parsecs).
\end{key}

\begin{key}{R\_MAX\_D}{362500.0}{}
The maximum distance to integrate to in the disk (in parsecs).
\end{key}

\begin{key}{R\_MAX\_T}{362500.0}{}
The maximum distance to integrate to in the thick disk (in parsecs).
\end{key}

\begin{key}{R\_MAX\_S}{362500.0}{}
The maximum distance to integrate to in the disk (in parsecs).
\end{key}

\begin{key}{R\_MIN}{0.0}{}
The distance to start integration from in the disk (in parsecs). This is
measured from the plane of the galaxy, not from the observer.
\end{key}

\begin{key}{SGC\_FILE}{m13.vbv}{(*)}
The file containing the colour--magnitude information for the spheroid giants.
\end{key}

\begin{key}{SLF\_FILE}{dacosta.vlf}{(*)}
The file containing the \lf\ for the spheroid.
\end{key}

\begin{key}{THICK}{FALSE}{}
Include a thick disk component in the model.
\end{key}

\begin{key}{TGC\_FILE}{dskm67.vbv}{(*)}
The file containing the colour--magnitude information for the thick disk 
giants.
\end{key}

\begin{key}{TLF\_FILE}{wielen.vlf}{(*)}
The file containing the \lf\ for the thick disk.
\end{key}

\begin{key}{USE\_MS}{FALSE}{}
Include a main-sequence \cmd\ in the calculations. See the descriptions
of the {\tt MSC\_FILE} and {\tt FMS\_FILE} keywords, and section 4.3 for 
more information.
\end{key}

\begin{key}{VERBOSE}{FALSE}{}
Print out all data arrays. This data is appended to the normal output of
\egm. The format used is described in section 3.3.
\end{key}

\begin{key}{X0}{23.74}{(*)}
The position of the field in arc minutes along the major axis.
\end{key}

\begin{key}{Y0}{0.00}{(*)}
The position of the field in arc minutes along the minor axis.
\end{key}

%%----------------------------------------------------------------------------|
\subsection{File Formats}

One of the key features of \egm\ is the  ability of the user  to specify a
luminosity function and a colour-magnitude diagram. If  you do use  your
own you may want to check that the right parameters are set.  All of the
defaults assume a $V$ luminosity function and a $B-V$  colour.  It is also
assumed that  every entry is a  floating point number (float or double).
\egm\ currently uses double precision in all floating point calculations.
For use with the {\tt -MS} option a file giving the fraction of stars on the
main sequence can also be specified. Any line beginning with a ``\#'' is 
considered a comment and is ignored.

\begin{table}[h]
\centering
\begin{tabular}{ccccc} \\
Luminosity Functions: & & Colour Magnitude Diagrams: & & Fraction on M-S: \\
M[1] \hspace{1cm} $\log_{10}$N[1] & \hspace{1cm} & M[1] \hspace{1cm} C[1] & \hspace{1cm} & M[1] \hspace{1cm} F[1] \\
... \hspace{1cm} ... & \hspace{1cm} & ... \hspace{1cm} ... & \hspace{1cm} & ... \hspace{1cm} ... \\
M[N] \hspace{1cm} $\log_{10}$N[N] & \hspace{1cm} & M[N] \hspace{1cm} C[N] & \hspace{1cm} & M[N] \hspace{1cm} F[N] 
\end{tabular}
\end{table}
\noindent where M, C, F and N are magnitudes, colours, fractions ($<1$) and 
numbers respectively.
 
%%----------------------------------------------------------------------------|
\subsection{Output Format}

\egm\ sends all output to the standard output stream -- it may therefore
be redirected to a file. The format used depends somewhat on the command
line options that have been set. The first part of the output is always
a header section, each line of which starts with a ``\#''. The data shown
here is basically a printout of all the parameters used to calculate
the model. A copy of the header section of the output is shown in Figure~1 (two
component model) and Figure~4 (three component model). The only real change
is the inclusion of the thick disk parameters in the latter. In both types the
value of the ``{\tt SPH}'' entry will change to ``{\tt C \& O}'' if the 
Caldwell \& Ostriker spheroid model has been used. The ``{\tt rek}'' entry 
will then be changed to ``{\tt rco}''. 

The section  beginning with ``{\tt Stars in COL}''  contains results  for  
the colour distribution in  the
order disk, spheroid  and total. The first line  is the number of stars,
then  the mean colours  of each distribution. Next  is the  fraction  of
stars on the main  sequence,  followed by  the fraction of giants  in
each one.  The number distribution  comes next: the  first 3 columns are
the magnitude  bin  ranges  and centre. In  the  next 6  columns, an ``A''
indicates  differential  counts (stars per magnitude bin), an ``N'' integral 
counts (cumulative counts down to that magnitude). The letters ``d'', ``s'' 
and ``t'' indicate  disk, spheroid and total counts respectively. 

Care should be taken when interpreting the
differential counts if the magnitude bin size ({\tt -bm} or 
{\tt MAG\_BIN}) is not equal to 0.0. In the default case 
the differential counts are in stars per magnitude bin (i.e. all the stars in
that bin between V1 and V2). Setting the magnitude
bin size to anything other than 0.0 will make \egm\ scale the counts to that
bin size by multiplying by ({\tt DM\_APP} / {\tt MAG\_BIN}).
Therefore the magnitude bin edges in columns 1 and 
3 will {\em not} be ``correct''. The bin centre (column 2) will be 
correct -- the {\tt -bm} option merely causes scaling of the differential 
number counts. The integral counts (``N'') are not affected by this option.

Figures~2 and 3 show the output format for the number counts and colour 
distributions and the verbose output ({\tt -v}) for the two component models. 
The latter two are, of course, only printed out if the appropriate options 
have been set. Figures~5 and 6 show the same things for the three component 
model. The last three figures (7, 8 and 9) are plots (produced using the \sm\ 
macros described in Appendix~B) of the results obtained using the default 
parameters.

The fields in the header section refer to the following (in the order you'll
read them in the file -- note that {\tt rmin} appears three times in the 
output -- it is possible that this parameter will be split into 3 (one for 
disk, thick disk and spheroid)):
%%--------------|
\begin{tabbing}
\rule{2cm}{0cm} \= \kill
x0  \> The position of the field in arc minutes along the major axis. \\
y0  \> The position of the field in arc minutes along the minor axis. \\
far \> Was the integration was carried through to the far side of the disk? \\
A   \> Field size in square degrees. \\
D   \> The distance to the galaxy in parsecs. \\
i   \> The inclination of the galaxy in degrees. \\
r0  \> Distance to galactic centre (in pc). \\
rpk \> Distance of peak of distribution from galactic plane (see \S 4.1) \\
mabrt \> Brightest apparent magnitude at which to begin integration.\\
madim \> Faintest apparent magnitude to carry integration down to.\\
dm    \> Apparent magnitude interval. \\
Mbin  \> Magnitude bin size for scaling differential counts. \\
Mbri  \> Brightest absolute magnitude for all the \lf s. \\
Mdim  \> Faintest absolute magnitude for all \lf s.\\
dM    \> The absolute magnitude interval. \\
mnbri \> The brightest magnitude for normalizing the \lf s.\\
Cerr  \> The colour error (in magnitudes). \\
Cint  \> Colour interval for calculation. \\
Cbin  \> The colour bin size for scaling colour distribution. \\
mndim \> The faintest magnitude for normalizing the \lf s.\\
maxC  \> Maximum colour of the colour distribution.\\
minC  \> Minimum colour of the colour distribution. \\
mbc   \> Faint apparent magnitude limit for colour distribution. \\
mfc   \> Bright apparent magnitude limit for colour distribution. \\
amode \> The absorption model used. \\
abs   \> Any arbitrary absorption included (e.g. from our Galaxy). \\
dska  \> The absorption added by looking through the disk.\\
a0    \> Scale height of the absorption. \\
a1    \> Absorption at the Galactic poles in the ``cosec'' absorption law. \\
a2    \> The $a_2$ parameter in the Sandage absorption law (\S4.2). \\
a3    \> The $a_3$ parameter in the Sandage absorption law (\S4.2). \\
den\_d \> The density normalization of the disk at rn\_d. \\
rn\_d  \> The normalization radius for the disk \lf. \\
psl\_d \> Scale length in the plane for the disk component (in pc). \\
gsh\_d \> Scale height of the disk giants (in pc). \\
dn\_d  \> Calculated disk normalization at r0. \\
dr\_d  \> The distance increment for the disk integration. \\
rmin   \> The minimum distance for integration. \\
rmaxd  \> The maximum distance for integration in the disk component. \\
dsklf  \> The disk luminosity function. \\
dgcmd  \> The disk giant colour magnitude diagram. \\
dmsfn  \> The main -- sequence colour magnitude diagram. \\
fmsfn  \> File containing the fraction of stars on main -- sequence. \\
den\_t \> The density normalization of the thick disk at rn\_t. \\
rn\_t  \> The normalization radius for the thick disk \lf. \\
psl\_t \> Scale length in the plane for the thick disk component (in pc). \\
gsh\_t \> Scale height of the disk giants (in pc). \\
dn\_t  \> Calculated thick disk normalization at r0. \\
dr\_t  \> The distance increment for the thick disk integration. \\
rmaxt  \> The maximum distance for integration in the thick disk. \\
tsklf  \> The thick disk luminosity function. \\
tgcmd  \> The thick disk giant colour magnitude diagram. \\
den\_s \> The density normalization of the spheroid at rn\_s. \\
rn\_s  \> The normalization radius for the spheroid \lf. \\
rek    \> The de Vaucouleurs radius of the spheroid (in pc). \\
es     \> The spheroid eccentricity/axis ratio. \\
dn\_s  \> Calculated spheroid normalization at r0. \\
dr\_s  \> The distance increment for the spheroid integration. \\
rmaxs  \> The maximum distance for integration in the thick disk. \\
sphlf  \> The spheroid luminosity function. \\
sgcmd  \> The spheroid giant colour magnitude diagram. \\
\end{tabbing}
%%--------------|

Additional information may be printed out at the end of the depending on
what options were set on the command line. The format of these extra data
blocks is one header line indicating the data type followed by two
columns of numbers. 

If ``verbose mode'' has been turned on (with the {\tt -v} option) 
\egm\ will print out data concerning the various distribution functions
that \egm\ has used in the model. They are not the same as the input files 
that were specified in
the parameter file. \egm\ uses spline interpolation on the input
\lf s and \cmd s and stores the resulting look--up--table in an array.
The  {\tt -v} option just tells \egm\ to print out these arrays.

First of these data blocks is the disk luminosity function
(header: DLF, format: magnitude  number),  then  the spheroid luminosity
function  (SLF) (format: magnitude  number).  Following  that are
three blocks  containing  the  colour-magnitude  diagrams  used:
main  sequence (MSC),  disk giant (DGC) and then  spheroid giant
(SGC) (format: magnitude colour). Finally the  fraction of stars
on the  main sequence (FMS) (format: magnitude  number) is given.

%%--------------|
%% Figure     1 |
%%--------------|
\begin{figure}[p]
\begin{center}
\begin{verbatim}
# EXTERNAL GALAXY MODEL: default.pm                          Components: 2 
#    x0 =     23.740    y0 =      0.000     A =      0.00100
#     D = 725000.000     i =     77.500   far =         NO  dska =      1.000
# mabrt =     16.000 madim =     26.000    dm =      0.500  Mbin =      0.000
#  Mbrd =     -6.000  Mbrs =     -3.000  Mdim =     16.500    dM =      0.050
#  Cerr =      0.100  Cint =      0.005  Cbin =      0.200
#  maxC =      3.000  minC =      0.000   mbc =     16.000   mfc =     26.000
#    dr =     25.000    r0 =   5006.702 amode =    Sandage   abs =      0.000
#    a0 =    100.000    a1 =      0.000    a2 =      0.165    a3 =      1.192
#  dn_d =    500.000  dn_t =      0.000  dn_s =      1.000    rn =   8000.000
#
# DSK:
# den_d =      1.723 psl_d =   5500.000 gsh_d =    250.000
# dsklf = wielen.vlf                  dgcmd = dskm67.vbv              
# dmsfn = NONE                        fmsfn = NONE                    
#
# SPH: R ^ (1/4)
# den_s =  9.196e-03   rek =   2670.000    es =      0.800
# sphlf = dacosta.vlf                   sgbfn = m13.vbv                 
#
#                     DSK           SPH            TOT
# Stars in COL   = 41438.973       2090.437      43529.410
# Mean Colours   =     1.196          0.839          1.179
# Star Fraction  =     0.952          0.048
# Giant Fraction =     1.000          1.000          1.000
#
\end{verbatim}
\end{center}
\caption{The header information for the two component (disk and spheroid)
model. If the Caldwell \& Ostriker spheroid model has been used the 
``{\tt SPH}'' entry will read {\tt C \& O} and {\tt rek} will be replaced by 
{\tt rco}.}
\end{figure}
%%--------------|

%%--------------|
%% Figure     2 |
%%--------------|
\begin{figure}[p]
\begin{center}
\begin{verbatim}
# NUM:
# V1    V     V2    Ad(V)    Nd(<V2)    As(V)    Ns(<V2)    At(V)    Nt(<V2)
16.50 16.75 17.00 0.000e+00 0.000e+00 0.000e+00 0.000e+00 0.000e+00 0.000e+00
17.00 17.25 17.50 0.000e+00 0.000e+00 0.000e+00 0.000e+00 0.000e+00 0.000e+00
17.50 17.75 18.00 0.000e+00 0.000e+00 0.000e+00 0.000e+00 0.000e+00 0.000e+00
18.00 18.25 18.50 6.381e-02 3.190e-02 0.000e+00 0.000e+00 6.381e-02 3.190e-02
18.50 18.75 19.00 1.028e+00 5.460e-01 0.000e+00 0.000e+00 1.028e+00 5.460e-01
					...
23.00 23.25 23.50 1.891e+03 1.848e+03 7.318e+01 5.732e+01 1.964e+03 1.905e+03
23.50 23.75 24.00 3.668e+03 3.682e+03 1.664e+02 1.405e+02 3.834e+03 3.822e+03
24.00 24.25 24.50 6.864e+03 7.114e+03 3.129e+02 2.970e+02 7.177e+03 7.411e+03
24.50 24.75 25.00 1.235e+04 1.329e+04 4.561e+02 5.250e+02 1.280e+04 1.381e+04
25.00 25.25 25.50 2.127e+04 2.392e+04 7.708e+02 9.104e+02 2.204e+04 2.483e+04
#
# COL:
#          B-V               Nd        Ns        NT
 0.0000  0.0025  0.0050     0.000     0.000     0.000
 0.0400  0.0425  0.0450     0.000     0.000     0.000
 0.0800  0.0825  0.0850     0.000     0.000     0.000
 0.1200  0.1225  0.1250     0.000     0.000     0.000
 0.1600  0.1625  0.1650     0.000     0.000     0.000
					...
 2.8400  2.8425  2.8450     5.690     0.000     5.690
 2.8800  2.8825  2.8850     4.572     0.000     4.572
 2.9200  2.9225  2.9250     3.571     0.000     3.571
 2.9600  2.9625  2.9650     2.666     0.000     2.666
 3.0000  3.0025  3.0050     1.865     0.000     1.865
\end{verbatim}
\end{center}
\caption{The format for the number counts and colour distribution for the 
two component (disk and spheroid) model.}
\end{figure}
%%--------------|

%%--------------|
%% Figure     3 |
%%--------------|
\begin{figure}[p]
\begin{center}
\begin{verbatim}
#
# V     DLF       SLF       FMS
-6.00  1.486e-08 2.645e-09 4.407e-01
-5.95  1.614e-08 2.912e-09 4.408e-01
-5.90  1.753e-08 3.206e-09 4.409e-01
-5.85  1.904e-08 3.530e-09 4.410e-01
-5.80  2.068e-08 3.886e-09 4.410e-01
		...
16.30  1.375e-02 1.375e-02 1.000e+00
16.35  1.373e-02 1.373e-02 1.000e+00
16.40  1.372e-02 1.372e-02 1.000e+00
16.45  1.372e-02 1.372e-02 1.000e+00
16.50  1.373e-02 1.373e-02 1.000e+00
#
# V     MSC       DGC        SGC
-6.00 -3.450e-01 3.132e+00 2.768e+00
-5.95 -3.441e-01 3.117e+00 2.753e+00
-5.90 -3.433e-01 3.102e+00 2.737e+00
-5.85 -3.424e-01 3.086e+00 2.722e+00
-5.80 -3.415e-01 3.071e+00 2.706e+00
		...
16.30  2.006e+00 2.006e+00 1.846e+00
16.35  2.012e+00 2.012e+00 1.852e+00
16.40  2.019e+00 2.019e+00 1.859e+00
16.45  2.025e+00 2.025e+00 1.865e+00
16.50  2.031e+00 2.031e+00 1.871e+00
\end{verbatim}
\end{center}
\caption{The information produced using verbose mode ({\tt -v}) for the two
component (disk and spheroid) model. In this case the {\tt -MS} option had
been used so the \fms\ and main-sequence \cmd\ files are displayed.}
\end{figure}
%%--------------|

%%--------------|
%% Figure     4 |
%%--------------|
\begin{figure}[p]
\begin{center}
\begin{verbatim}
# EXTERNAL GALAXY MODEL: default.pm                       Components: 3 
#    x0 =     23.740    y0 =      0.000     A =      0.00100
#     D = 725000.000     i =     77.500   far =         NO  dska =      1.000
# mabrt =     16.000 madim =     26.000    dm =      0.500  Mbin =      0.000
#  Mbrd =     -6.000  Mbrs =     -3.000  Mdim =     16.500    dM =      0.050
#  Cerr =      0.100  Cint =      0.005  Cbin =      0.200
#  maxC =      3.000  minC =      0.000   mbc =     16.000   mfc =     26.000
#    dr =     25.000    r0 =   5006.702 amode =    Sandage   abs =      0.000
#    a0 =    100.000    a1 =      0.000    a2 =      0.165    a3 =      1.192
#  dn_d =    500.000  dn_t =     10.000  dn_s =      1.000    rn =   8000.000
#
# DSK:
# den_d =      1.723 psl_d =   5500.000 gsh_d =    250.000
# dsklf = wielen.vlf                  dgcmd = dskm67.vbv              
# dmsfn = NONE                        fmsfn = NONE                    
#
# THK:
# den_t =  3.447e-02 psl_t =   5500.000 gsh_t =   1300.000
# thklf = wielen.vlf                  tgbfn = dskm67.vbv              
#
# SPH: R ^ (1/4)
# den_s =  9.196e-03   rek =   2670.000    es =      0.800
# sphlf = dacosta.vlf                 sgbfn = m13.vbv                 
#
#                     DSK            THK           SPH            TOT
# Stars in COL   = 41438.973       2556.082      2090.437      46085.492
# Mean Colours   =     1.196          1.224         0.839          1.181
# Star Fraction  =     0.899          0.055         0.045
# Giant Fraction =     1.000          1.000         1.000          1.000
#
\end{verbatim}
\end{center}
\caption{The header information for the three component (disk, thick disk
and spheroid) model. If the Caldwell \& Ostriker spheroid model has been used 
the ``{\tt SPH}'' entry will read {\tt C \& O} and {\tt rek} will be replaced 
by {\tt rco}.}
\end{figure}
%%--------------|

%%--------------|
%% Figure     5 |
%%--------------|
\begin{figure}[p]
\begin{center}
\begin{verbatim}
# NUM:
# V1    V     V2    Ad(V)    Nd(<V2)    At(V)    Nt(<V2)
16.50 16.75 17.00 0.000e+00 0.000e+00 0.000e+00 0.000e+00
17.00 17.25 17.50 0.000e+00 0.000e+00 0.000e+00 0.000e+00
17.50 17.75 18.00 0.000e+00 0.000e+00 0.000e+00 0.000e+00
18.00 18.25 18.50 6.381e-02 3.190e-02 1.320e-03 6.599e-04
18.50 18.75 19.00 1.028e+00 5.460e-01 2.538e-02 1.335e-02
					...
23.00 23.25 23.50 1.891e+03 1.848e+03 1.111e+02 1.077e+02
23.50 23.75 24.00 3.668e+03 3.682e+03 2.179e+02 2.166e+02
24.00 24.25 24.50 6.864e+03 7.114e+03 4.129e+02 4.231e+02
24.50 24.75 25.00 1.235e+04 1.329e+04 7.526e+02 7.994e+02
25.00 25.25 25.50 2.127e+04 2.392e+04 1.315e+03 1.457e+03

# V1    V     V2    As(V)    Ns(<V2)    AT(V)    NT(<V2)
16.50 16.75 17.00 0.000e+00 0.000e+00 0.000e+00 0.000e+00
17.00 17.25 17.50 0.000e+00 0.000e+00 0.000e+00 0.000e+00
17.50 17.75 18.00 0.000e+00 0.000e+00 0.000e+00 0.000e+00
18.00 18.25 18.50 0.000e+00 0.000e+00 6.513e-02 3.256e-02
18.50 18.75 19.00 0.000e+00 0.000e+00 1.054e+00 5.594e-01
					...
23.00 23.25 23.50 7.318e+01 5.732e+01 2.075e+03 2.013e+03
23.50 23.75 24.00 1.664e+02 1.405e+02 4.052e+03 4.039e+03
24.00 24.25 24.50 3.129e+02 2.970e+02 7.590e+03 7.834e+03
24.50 24.75 25.00 4.561e+02 5.250e+02 1.356e+04 1.461e+04
25.00 25.25 25.50 7.708e+02 9.104e+02 2.336e+04 2.629e+04
#
# COL:
#          B-V               Nd        Nt        Ns        NT
 0.0000  0.0025  0.0050     0.000     0.000     0.000     0.000
 0.0400  0.0425  0.0450     0.000     0.000     0.000     0.000
 0.0800  0.0825  0.0850     0.000     0.000     0.000     0.000
 0.1200  0.1225  0.1250     0.000     0.000     0.000     0.000
 0.1600  0.1625  0.1650     0.000     0.000     0.000     0.000
			...
 2.8400  2.8425  2.8450     5.690     0.410     0.000     6.100
 2.8800  2.8825  2.8850     4.572     0.330     0.000     4.901
 2.9200  2.9225  2.9250     3.571     0.258     0.000     3.828
 2.9600  2.9625  2.9650     2.666     0.192     0.000     2.858
 3.0000  3.0025  3.0050     1.865     0.135     0.000     2.000
\end{verbatim}
\end{center}
\caption{The format for the number counts and colour distribution for the 
three component (disk, thick disk and spheroid) model.}
\end{figure}
%%--------------|

%%--------------|
%% Figure     6 |
%%--------------|
\begin{figure}[p]
\begin{center}
\begin{verbatim}
#
# V    DLF       TLF       SLF
-6.00 1.486e-08 1.486e-08 2.645e-09
-5.95 1.614e-08 1.614e-08 2.912e-09
-5.90 1.753e-08 1.753e-08 3.206e-09
-5.85 1.904e-08 1.904e-08 3.530e-09
-5.80 2.068e-08 2.068e-08 3.886e-09
...
16.30 1.375e-02 1.375e-02 1.375e-02
16.35 1.373e-02 1.373e-02 1.373e-02
16.40 1.372e-02 1.372e-02 1.372e-02
16.45 1.372e-02 1.372e-02 1.372e-02
16.50 1.373e-02 1.373e-02 1.373e-02
#
# V    DGC       TGC       SGC
-6.00 3.132e+00 3.132e+00 2.768e+00
-5.95 3.117e+00 3.117e+00 2.753e+00
-5.90 3.102e+00 3.102e+00 2.737e+00
-5.85 3.086e+00 3.086e+00 2.722e+00
-5.80 3.071e+00 3.071e+00 2.706e+00
...
16.30 -6.167e+02 -6.167e+02 -1.818e+01
16.35 -6.243e+02 -6.243e+02 -1.846e+01
16.40 -6.319e+02 -6.319e+02 -1.873e+01
16.45 -6.397e+02 -6.397e+02 -1.901e+01
16.50 -6.474e+02 -6.474e+02 -1.929e+01
\end{verbatim}
\end{center}
\caption{The information produced using the {\tt -v} option for the three 
component (disk, thick disk and spheroid) model. The main sequence was not
included.}
\end{figure}
%%--------------|

%%--------------|
%% Figure     7 |
%%--------------|
\begin{figure}[p]
\centerline{\psfig{figure=figure7.eps}}
\caption{The differential number counts for the default model.}
\end{figure}

%%--------------|
%% Figure     8 |
%%--------------|
\begin{figure}[p]
\centerline{\psfig{figure=figure8.eps}}
\caption{The integral number counts for the default model.}
\end{figure}

%%--------------|
%% Figure     9 |
%%--------------|
\begin{figure}[p]
\centerline{\psfig{figure=figure9.eps}}
\caption{The colour distribution produced by the default model.}
\end{figure}
%%--------------|

%%----------------------------------------------------------------------------|
%% MODEL DESCRIPTION                                                          |
%%----------------------------------------------------------------------------|

\newpage
\section{Model Description}

%%------------------------------------|
\subsection{Geometrical Considerations}

In order to use \egm\ properly it is necessary to know what the parameters
describing positions and angles actually refer to. The first step is to
convert the Right Ascension and Declination of a field to rectangular
coordinates, as shown in figure~10. The $x$ and $y$ axes are defined to
be along the major and minor axes of the galaxy. The $x$-axis is inclined
to the R.A. axis by the position angle $P$. It is a simple task to transform
the coordinates of G from $(\alpha,\delta)$ to $(x,y)$:
\[
x = \alpha \cos P - \delta \sin P 
\]
\[
y = \alpha \sin P + \delta \cos P
\]
%%--------------|
%% Figure    10 |
%%--------------|
\begin{figure}[b]
\centerline{\psfig{figure=figure10.eps,height=8cm}}
\caption{Converting the R.A. and Dec. of the field at G to {\sf x} and
{\sf y} coordinates. These are defined to lie along the major and minor
axes of the galaxy, respectively.}
\end{figure}

\egm\ calculates the number counts using a variation of the Bahcall \&
Soneira Model. In practice this means imagining that the observer is at
the field (G, in figure~10) and calculating the counts back along the line
of sight to the observer. This requires the ``galactic latitude and
longitude'' of the observer as seen from G. Figures~11 and 12 show the
geometry of the situation. In figure~12 an observer at O looks at a field P
in a galaxy whose plane (G) is inclined at am angle $i$ to the plane of the
sky (S). The angle $b$ is the ``galactic latitude'' of O as seen from P.
Provided $D >> x$ then $d \approx D$ and $e \approx 0$. Thus we may assume
that $b = 90^\circ - i $. For the inclination and distance of M31 and a field
where $x = 55 {\rm Kpc}$ this is true to within 0.5\%.
%%--------------|
%% Figure    11 |
%%--------------|
\begin{figure}[p]
\centerline{\psfig{figure=figure11.eps,width=16cm,angle=-90}}
\caption{A simplified view of the geometry of the situation. The plane of
the galaxy (G) is inclined at an angle $i$ to the plane of the sky (S)
(as seen from O). Assuming $ x << D$ we have $b = 90^\circ - i$.}
\end{figure}
%%--------------|
%%--------------|
%% Figure    12 |
%%--------------|
\begin{figure}[p]
\centerline{\psfig{figure=figure12.eps,width=16cm}}
\caption{As seen by the observer the field Q on the sky projects to P in the
galaxy. $r$ and $l$ can be found by geometrical analysis.}
\end{figure}
%%--------------|

The model now needs to know r, the distance of the field P from the
centre of the galaxy, and $\ell$ the ``galactic longitude''. Figure~12 shows
the geometry in this case. It is necessary to define a coordinate
system on the plane of the sky -- x and y in the figure. The x-axis is
defined to lie along the line of nodes of the intersection of the galaxy
with the plane of the sky. So we give the model the (x,y) coordinates of the
field Q (on the sky, which projects to P in the galaxy). Then
\[
r = \rho \left( \cos^2 \phi + \sec^2 i \sin^2 \phi \right) ^ {\frac{1}{2}},
\]
\[
\theta = \arctan \left( \frac{\tan \phi}{\sec i} \right),
\]
\[
\ell = 90^\circ - \theta.
\]

%%--------------------------|
\subsection{Component Models}

There are currently only two types of components in \egm\ -- the thin, 
exponential
disk and the spheroidal halo. The thick disk is functionally equivalent to
the thin disk model. The functional forms discussed below have been
hard--coded into the program and cannot be changed without rewriting the
appropriate subroutines. The parameters in these functional forms can,
of course, be set by the user.

%%----------------------|
\subsubsection*{The Disk}

The density  distribution of  the disk stars in the Bahcall and  Soneira
model is represented by the following exponential function:
\[ 
\rho_d(r,M) = e^{\left[\frac{-z}{H} - \frac{x-R_0}{h} \right]}
\] 
where the $H$ scale height of the disk and $h$ is the scale length. 
$R_0$ is the distance of the field from the galactic centre (in pc).
 
%%--------------------------|
\subsubsection*{The Spheroid}

de Vaucouleurs (1959,  in  Handbuch  der Physik, Vol. 53,  ed. S. Flugge
(Berlin, Springer--Verlag), p. 311) found  that the projected  brightness
distribution of ellipticals was given by the famous $r^\frac{1}{4}$ law:
\[
\log \frac{I(r)}{I(r_e)} = -3.3307 \left( \left(\frac{r}{r_e}\right)^
{\frac{1}{4}} - 1 \right)
\]
where  $r_e$  is the spatial distance that  projects to an angle containing
half the total luminosity. Young (1976, AJ, 81, 807) gives an asymptotic
approximation for the spatial density of stars  which leads to this form
of projection:
\[
\rho_s(r) \approx \frac{e^{-b\left(\frac{r}{r_e}\right)}}
{2\left(\frac{r}{r_e}\right)^3}
\left( \frac{\pi}{8 b \left(\frac{r}{r_e}\right)} \right)^{\frac{1}{2}}
\]
where $b = 7.6692$.

%%------------------------|
\subsubsection*{Absorption}

Three different models for galactic absorption are currently supported -- no
absorption, the ``cosec'' law and the Sandage absorption law, the latter being
the default model. The absorption model can be specified on the command line
(the {\tt -a} option) or in the parameter file with the keyword {\tt AMODE}. 
In both cases the words ``none'', ``cosec'' and ``Sandage'' can be used to 
choose the desired model. Only the first letter of the word (which may be 
in upper or lower case) is significant.

Each absorption mode needs various parameters (except the no absorption
case!). The cosecant law calculates absorption based on the formula
\[
A(b) = a_1(90^\circ) \csc b
\]
where $a_1(90^\circ) = 0.15$ magnitudes in $V$ (the default) and 
$A_V = 0.75 A_B$. This coefficient can be specified in the parameter file
with the {\tt A\_1} keyword. You will have to decide on the value of this 
parameter for other filters on your own.

The Sandage absorption model (Sandage, 1972, ApJ, 178, 1) in the $V$ band is
\[
\begin{array}{lll}
A(b) = & a_2 ( a_3 - \tan b) \csc b & |b| \leq 50^\circ \\
A(b) = & 0 & |b| > 50^\circ
\end{array}
\]
where for the $V$ filter $a_2 = 0.165$ and $a_3 = 1.192$. In the $B$ filter
$A_B = 1.33 A_V$. To the best of my knowledge this formula is only defined
for the $V$ and $B$ filters. The parameters $a_2$ and $a_3$ can be 
reset in the parameter file with the {\tt A\_2} and {\tt A\_3} keywords 
respectively.

Finally the absorption in magnitudes at distance $R$ is calculated from
\[
A(R) = A(b) \left[ 1 - e^{(- \sin b / a_0) R} \right]
\]
where $a_0$ is the scale height of the absorbing material. The default is
100 pc, and the appropriate keyword is {\tt A\_0}. It is assumed there is no
variation with distance from the galactic centre, only an exponential
variation perpendicular to the plane.

Two other absorption parameters may be defined. The first is simply a 
number (in magnitudes) added to the calculations (use the {\tt ABS} keyword 
or the {\tt -A} option) -- this may simulate absorption in our Galaxy. The
second (using the {\tt DSK\_ABS} keyword) only applies if the {\tt FAR\_SIDE} 
option is set -- this number is added to all calculations that occur of the 
far side of the disk. This simulates looking through a thin dust layer in the 
plane of the galaxy.

%%-----------------------------------|
\subsubsection*{Density Normalization}

One item of program input that needs careful consideration is the density
normalization to choose for the disk, thick disk and spheroid components.
This is done with the {\tt -dnd}, {\tt -dnt} and {\tt -dns} options on the 
command line or the corresponding {\tt DND}, {\tt DNT} and {\tt DNS}
parameters. The format is: {\tt density@radius} where {\tt density} is the 
density on stars per cubic parsec at a distance of {\tt radius} (in the 
plane of the galaxy). 

The \lf\ is then normalized to {\tt density} stars per cubic parsec, between
the magnitudes given by the {\tt MN\_BRI} and {\tt MN\_DIM} parameters.

The distance $r_0$ of the field from the centre of the
galaxy is then calculated and the actual density normalizations at that point
are found by re--scaling from {\tt radius}. For example, the default field is 
at 5006.702 pc from the centre of the galaxy which gives a disk density of 1.723 and a spheroid density of \sci{9.196}{-3}. 

%%----------------------------------------------------------------------------|
\subsection{Distribution Functions}

%%----------------------------------|
\subsubsection*{Luminosity Functions}

In this release of \egm, three files are given for the luminosity function
of the disk (``analytic.vlf'', ``wielen.vlf'', and ``dacosta.vlf''). 
The first of these, the pure analytic \lf, is given by equation (1) of 
Bahcall and Soneira 1980 (ApJS, 44, 73):
\[
\begin{array}{lll}
\phi(M) = & \frac{n_\ast 10^{\beta(M-M_\ast)}}
{\left( 1 + 10^{-(\alpha - \beta)\delta(M-M_\ast)}\right)^{\frac{1}{\delta}}}
& M_b \leq M \leq M_c \nonumber \\
\phi(M) = & \phi(M_c) & M_c \leq M \leq M_d  \\
\phi(M) = & 0 & M \leq M_b {\rm ~or~~} M \geq M_d
\end{array}
\]
where, for the $V$ band, $n_\ast = 4.03 \times 10^{-3}$, $M_\ast = +1.28$, 
$\alpha = 0.74$, $\beta = 0.04$, $\frac{1}{\delta} = 3.40$, $M_b = -6$, 
$M_c = +15$ and $M_d = +19$.
 
This has been modified to include the ``Wielen dip'' (Wielen, 1974, in
Highlights of Astronomy, Vol. 3, ed. G. Contopoulos (Dordrecht, Reidel), 
p. 365) in the file ``wielen.vlf'' and further modified to include the so--
called ``globular  cluster feature''  (DaCosta, 1982, AJ, 87, 990) in  the
file ``dacosta.vlf''. As discussed in Bahcall and Soneira (1980)  these can
probably be used for the spheroid \lf\ without too much of a problem.

%%----------------------------------------|
\subsubsection*{Colour--Magnitude Diagrams}

\egm\ uses \cmd s only to calculate the expected colour distribution of the
field in question. Thus if you are interested in say, number counts in $I$
but have no $V-I$ \cmd, you could run \egm\ with an $I$ band \lf\ and ignore 
the colour distribution that \egm\ produces.

Several \cmd s are provided with \egm\ in the standard library. The $B-V$ ones
come mainly from the functions supplied with the FORTRAN version of the
Bahcall \& Soneira code. The $V-I$ are take from Da~Costa \& Armandroff 
(1990, AJ, 100, 162)

%%----------------------------------------|
\subsubsection*{Fraction of stars on the main--sequence}
If the {\tt -MS} option has been set \egm\ needs to know the fraction of
stars on the main sequence as a function of absolute magnitude. This is
so it can weight the giant branch and main sequence number counts
accordingly. Mamon \& Soneira, 1982 (ApJ, 255, 181) give the following
analytical form for the number of stars on the main sequence:
\[
\begin{array}{lll}
f(M) = & C e^{\alpha(M+\beta)^{\gamma}} & M < M_a \\
f(M) = & 1 & M \geq M_a
\end{array}
\]
where for the $V$ band, $C=0.44$, $\alpha = 1.5\times10^{-4}$, $\beta=8.0$,
$\gamma=3.5$ and $M_a = 3.7$. 

%%----------------------------------------------------------------------------|
\subsection{Program Function Details}

This section describes how \egm\ produces a model. It is reasonably 
detailed and is {\em not} required reading. In some cases it refers to 
what the computer is actually doing, so if you're really interested
you may want to have a copy of the source code close at hand.
The discussion also assumes that all the files \egm\ needs are available
and that it won't crash!

\subsubsection*{Initialization}
The first thing \egm\ does is initialize all the parameters to default
values. Then it scans the command line for the name of the parameter file and
reads it one line at a time. Each line is parsed for a keyword and this
is checked against and internal list of valid keywords.
If it comes across an invalid keyword it will stop,
otherwise it will overwrite the default value of that parameter with the
value in the parameter file. 

It then scans the command line for options and updates the appropriate 
parameters, checking for invalid options and overwriting the appropriate
parameter values. Command line options are scanned last and will therefore
overwrite values given in the parameter file.

Next \egm\ initializes various look-up-tables (LUTs) that it needs. These are
arrays for the \lf, \cmd\ (and \fms\, if required) data. From 
it's parameters it determines the size of each array needed and reads in 
the corresponding data file. It performs
spline interpolation on this data to fill in the array. Several arrays are
needed -- one each for the disk and spheroid \lf s, two for the disk giant 
and spheroid giant color magnitude diagrams and possibly two for the main
sequence \cmd\ and for the \fms\ data. Corresponding arrays will be needed for 
the thick disk if it is to be included. Other arrays used to hold the results 
of the integration (number counts and colour distribution) are also set up at 
this time but are merely initialized to zero.

\subsubsection*{Integration Procedure}
\egm\ now calculates the ``projection'' parameters -- the effective
galactic latitude and longitude based on the input parameters as described
in section 4.1.

\egm\ begins with the disk component. Starting at R\_MIN it moves along the 
line of sight in steps of DR. It then calculates the size of the volume 
element, its distance from the observer (not from the galaxy) and the 
absorption (if any). The distances of the element above and along the plane
of the galaxy are calculated to enable the density at that point to be found.

The program then steps through the range of absolute magnitudes (defined
by M\_DIM and M\_BRI). For each of these steps it calculates the apparent
magnitude -- if this is outside the desired range (MA\_DIM and MA\_BRT)
it procedes to the next step, otherwise it looks up the number of
stars at that absolute magnitude in the disk luminosity function LUT.
If the main-sequence is being included the fraction of stars on the
giant branch and main-sequence are found using the \fms\ LUT.
The number of stars is then found by multiplying
together the volume, the \lf\ at that (absolute) magnitude, the
\fms\ value and the density distrbution, evaluated at that position.

The effect on the colour distribution is found by looking up the colour
corresponding to that absolute magnitude on the \cmd\ LUTs. The count
for that particular colour is filled in and various other running totals
(for the mean colour etc.) are computed.

The program then loops back for the next magnitude step in the \lf.
When that loop is finished it proceeds to the next distance step and so on.
Integration is terminated when all loops have been finished or when
the following condition is met. The total counts accumulated over the
\lf\ loop are compared to the running grand total of counts. If this 
sub-total is less than C\_FAC times the grand total then an insignificant
amount of stars are being added and the integration stops.

The calculation of the spheroid counts proceeds in pretty much the same way,
except for one slight difference. The spheroid generally extends much farther
than the disk, therefore because of the larger numbers involved the
distance steps are done in logarithmic intervals to speed up the calculation.
Tests have shown that this does not introduce any appreciable difference to
calculating in linear steps.

The calculations for the far side of the galaxy proceed in an identical
manner, except for the inclusion of extra absorption and the fact that as
distances from the galaxy increase, so do the distances from the observer.

After this all that remains is some ``tidying up''. The computed colour
distribution is convolved with a Gaussian error distribution to obtain the 
predicted colour distribution. Various other ``colour'' statistics are also
worked out. The total of the number counts are found, as is the integral
number counts for each component.
The results are then written out on the standard output stream.

\end{document}
