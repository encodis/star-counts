%%----------------------------------------------------------------------------|
%% GMU USERS GUIDE                                                            |
%%                                                                            |
%% P. J. C. Hodder                                                            |
%%----------------------------------------------------------------------------|

\documentclass[11pt,twoside]{article}
\input{gmu_macros.tex}

\begin{document}  

%%----------------------------------------------------------------------------|
%% TITLE PAGE                                                                 |
%%----------------------------------------------------------------------------|

\thispagestyle{empty}
\noindent
\framebox[16cm]{\parbox{16cm}{
{\begin{center}
\huge \bf 
A Users Guide to the 

\rule{0mm}{10mm}

A Galaxy Model Utilities

\rule{0mm}{10mm}

Version 1.2

\end{center}}}}

\rule{0mm}{10mm}

\noindent
\framebox[16cm]{\parbox{16cm}{
{\begin{center}
\large \bf
Contents
\end{center}}
\large
\begin{tabbing}
\rule{2cm}{0cm} \= \rule{1cm}{0cm} \= \kill
\> 1  \> Introduction \\ \\
\> 2  \> A Brief Review of the GMU \\ \\
\> 3  \> Make Analytic LFs: mkalf \\ \\
\> 4  \> Make New CMDs: mkcmd \\ \\
\> 5  \> Fraction of Stars: mkfms \\ \\
\> 6  \> LFs from MLRs: mkmlf \\ \\
\> 7  \> Normalize LFs: mknlf \\ \\
\> 8  \> Spline Fitting: mkspl \\ \\
\> 9  \> Make Linear LFs: mkxlf \\ \\
\> 10 \> Comparing to data: egmcmp \\ \\
\> 11 \> Adding it up: addegm \\ \\
\> A \> Installing the GMU \\ \\
\end{tabbing}
}}

%%----------------------------------------------------------------------------|
%% INTRODUCTION                                                               |
%%----------------------------------------------------------------------------|

\newpage
\setcounter{page}{1}
\section{Introduction}

This document describes the installation and use of a collection of software
tools known as the Galaxy Model Utilities (or \gmu\ for short). They are 
designed to operate in concert with the {\sc bsm} (Bahcall -- Soneira Model)
and {\sc egm} (External Galaxy Model) programs. To function properly these 
two programs require luminosity functions (\lf s) and colour--magnitude
diagrams (\cmd s) as input data. The \gmu\ aid in the construction and 
preparation of these files.

%%----------------------------------------------------------------------------|
%% A BRIEF REVIEW                                                             |
%%----------------------------------------------------------------------------|

\section{A Brief Review}

The \gmu\ are relatively simple to learn -- each of the 9 programs basically
does one thing only, thus drastically reducing the number of command line 
options. In fact for several of the \gmu\ there is only one way to use the
program! In most cases the \gmu\ operate on and produce data in two column 
ASCII format. The first column is usually the magnitude; the second is either 
a number or a colour (depending on the program). The columns must be separated 
by spaces, not commas. Any line beginning with a ``\#'' is considered a comment
and is skipped on input. The majority of the \gmu\ act as filters -- the 
filename can be entered on the command line or the standard input stream can be
piped to the program. All output is to ``standard out'' so the output must 
be redirected to a file or piped to another program. All the \gmu\ (except
\mkspl , \addegm , and \egmcmp\ write out a comment line first -- the contents
of this comment is the command line used to generate that output. This may help
you keep track of ``who created what''!

In the descriptions below an option in brackets ({\tt []}) is optional and may
be left out. Two options separated by a ``{\tt |}'' indicates that either one
or the other may be used.

Many of these utilities perform spline fits to the data for purposes of 
interpolation. Thus the input data must be monotonic increasing (the first
column, that is).

A short help for all the \gmu\ can be obtained by typing the program's name
with no arguments. All the programs return an error level of 0 upon successful
completion and 1 upon failure (of any sort).

\vspace{10mm}
\centerline{DISCLAIMER}
\nin As far as I know these programs work as described in this manual. They are
provided ``as is'' with no warranty or money back guarantee. If they don't
work for you then that's the way it is. Accept the fact that the Universe hates
you. Note, however, that the author can be bribed with cookies, money, fast 
cars and loose women to fix problems as they arise. 

\vspace{10mm}
\nin Any resemblance of these programs to any real science, programming 
ability or general usefulness is purely coincidental. 

%%----------------------------------------------------------------------------|
%% PROGRAM DESCRIPTIONS                                                       |
%%----------------------------------------------------------------------------|

%%----------------------------------------------------------------------------|
%% MKALF                                                                      |
%%----------------------------------------------------------------------------|

\newpage
\section{The \mkalf\ Utility}

\mkalf\ is an acronym for ``MaKe Analytic Luminosity Function''. The analytic
form is given in Bahcall \& Soneira (1980, ApJS, 44, 73) as shown below:
\[
\begin{array}{lll}
\phi(M) = & \frac{n_\ast 10^{\beta(M-M_\ast)}}
{\left( 1 + 10^{-(\alpha - \beta)\delta(M-M_\ast)}\right)^{\frac{1}{\delta}}}
& M_b \leq M \leq M_c \nonumber \\
\phi(M) = & \phi(M_c) & M_c \leq M \leq M_d  \\
\phi(M) = & 0 & M \leq M_b {\rm ~or~~} M \geq M_d
\end{array}
\]

Values of these parameters for the six default filters are:

\begin{table}[h]
\centering
\begin{tabular}{clllllllll}
  & $n_\ast$ & $M_\ast$ & $\alpha$ & $\beta$ & $\frac{1}{\delta}$ & $M_b$ & $M_c$& $M_d$ & $N(M_c)$ \\ \hline
B & 0.00255 & 2.20 & 0.6   & 0.05  & 2.3  & -6.0 & +15.0 & +18.0 & 1.0 \\ 
V & 0.00403 & 1.28 & 0.74  & 0.04  & 3.40 & -6.0 & +15.0 & +19.0 & 0.01414 \\
R & 0.00480 & 1.40 & 0.74  & 0.045 & 3.0  & -6.0 & +14.0 & +13.0 & 1.0 \\
I & 0.0175  & 0.85 & 0.81  & 0.01  & 5.20 & -6.0 & +14.0 & +11.0 & 1.0 \\
J & 0.0250  & 0.6  & 0.8   & 0.01  & 5.65 & -5.0 & +11.0 & +10.0 & 1.0 \\
K & 0.0290  & 2.0  & 0.625 & 0.01  & 4.0  & -5.0 & +9.0  & +9.0  & 1.0 \\ 
\hline
\end{tabular}
\end{table}

%%----------------------------------------------------------------------------|
\subsection*{Usage}

\begin{verbatim}
mkalf filter|file bright faint delta
\end{verbatim}

\nin
The order of the command line arguments cannot be changed. They refer to:
\medskip

\begin{clo}{filter|file}
This specifies the band in which the \lf\ is to be constructed. The bands 
currently supported are $B$, $V$, $R$, $I$, $J$, and $K$. If the argument is 
not one of these then \mkalf\ will try to read the parameters from a file.
\end{clo}

\begin{clo}{bright}
The brightest absolute magnitude for output. Magnitudes brighter than $M_b$ are
not printed.
\end{clo}

\begin{clo}{faint}
The faintest absolute magnitude for output. Magnitudes fainter than $M_d$ are
not printed.
\end{clo}

\begin{clo}{delta}
The absolute magnitude interval for output.
\end{clo}

%%----------------------------------------------------------------------------|
\subsection*{Notes}

There are six bands pre-programmed into \mkalf : $B$, $V$, $R$, $I$, $J$,
and $K$. The band name is case sensitive -- $v$ is not the same as $V$. If the 
``filter'' is not one of these letters then the program will try to find a 
file of that name, and use data in this file to get the correct parameters for 
the analytic expression. (This means that a file called ``V'' will not be 
read, of course.) The format of this one line file is just nine, whitespace
separated floating point numbers:

\begin{tabbing}
\rule{1cm}{0cm} \= \rule{1cm}{0cm} \= \rule{1cm}{0cm} \= \rule{1cm}{0cm} \= \rule{1cm}{0cm} \= \rule{1cm}{0cm} \= \rule{1cm}{0cm} \= \rule{1cm}{0cm} \= \rule{1cm}{0cm} \= \kill
$n_\star$ \> $M_\star$ \> $\alpha$ \> $\beta$ \> $\frac{1}{\delta}$ 
\> $M_b$ \> $M_c$ \> $M_d$ \> $N(M_c)$
\end{tabbing}

The output is printed on the standard output -- redirection to a file is 
recommended. The data produced by \mkalf\ is, of course, compatible with 
{\sc bsm} and {\sc egm} (two column format, the first being magnitude, the
second being the $\log_{10}$ of the number at that magnitude.

%%----------------------------------------------------------------------------|
%% MKCMD                                                                      |
%%----------------------------------------------------------------------------|

\newpage
\section{The \mkcmd\ Utility}

\mkcmd\ is a tool to aid in the production of \cmd s. The functionality is 
fairly simple, and could probably be accomplished with {\tt awk} or {\tt perl}
scripts, but it is included in the package for completeness. There are two
principle modes of operation. The first merely adds a colour excess to the
colours in the input \cmd. The output is then the shifted \cmd. The second 
requires two \cmd s, one for the M-S, the other for the RGB, and the turn--off
magnitude (and optionally a colour excess to be added). The output is the
RGB \cmd\ above the turn--off magnitude, and the MS \cmd\ below that. If 
present the colour excess is added.

%%----------------------------------------------------------------------------|
\subsection*{Usage}

\begin{verbatim}
mkcmd ce cmd

mkcmd to [ce] mscmd gbcmd
\end{verbatim}

\nin
The order of the command line arguments cannot be changed. They refer to:
\medskip

\begin{clo}{to}
The magnitude of the turn off.
\end{clo}

\begin{clo}{ce}
The colour excess (or blueshift).
\end{clo}

\begin{clo}{cmd, mscmd, gbcmd}
Files containing \cmd s -- in the first case any \cmd\ can be specified. In 
the second the order is important (MS first, RGB second).
\end{clo}

%%----------------------------------------------------------------------------|
\subsection*{Notes}

\mkcmd\ works on standard {\sc bsm} and {\sc egm} \cmd\ files (two column
format, magnitude and colour) and outputs data in the same format. \mkcmd\
replaces part od the functionality that was built into {\sc bsm}: in the
second mode of operation the output \cmd\ is a combination of of the M-S
and RGB \cmd s. For magnitudes brighter than the turn--off the RGB \cmd\ is
used. Fainter than this the M-S \cmd\ is used. The limits on the output \cmd\
are the brightest and faintest magnitudes of the two input \cmd s. The 
magnitude interval is the difference between the first and second magnitudes
of the M-S \cmd. You may want to run the input magnitudes through \mkspl\ if
this is a bad choice.

%%----------------------------------------------------------------------------|
%% MKFMS                                                                      |
%%----------------------------------------------------------------------------|

\newpage
\section{The \mkfms\ Utility}

\mkfms\ makes a file containing the run of fraction of stars on the main
sequence as a function of absolute magnitude. It is calculated according
to the equation:
\[
\begin{array}{lll}
f(M) = & C e^{\alpha(M+\beta)^{\gamma}} & M < M_a \\
f(M) = & 1 & M \geq M_a
\end{array}
\]

Values of these parameters for the six default filters are:

\begin{table}[h]
\centering
\begin{tabular}{clllll}
  & $C$    & $\alpha$ & $\beta$ & $\gamma$ & $M_a$ \\ \hline
B & 0.53   & 2.4e-31  & 75.0    & 16.0     & 4.6 \\
V & 0.44   & 1.5e-4   & 8.0     & 3.5      & 3.7 \\
R & 0.31   & 6.5e-4   & 7.5     & 3.2      & 2.9 \\
I & 0.08   & 0.37     & 4.4     & 1.0      & 2.4 \\
J & 0.0069 & 2.0      & 4.0     & 0.5      & 2.2 \\
K & 0.0047 & 2.5      & 3.0     & 0.5      & 1.6 \\ \hline
\end{tabular}
\end{table}

%%----------------------------------------------------------------------------|
\subsection*{Usage}

\begin{verbatim}
mkfms filter|file bright faint delta
\end{verbatim}

\nin
The order of the command line arguments cannot be changed. They refer to:
\medskip

\begin{clo}{filter}
This argument specifies the filter for which the \fms\
file is to be made. The filters currently supported are $B$, $V$, $R$, $I$, $J$
and $K$. If the argument is not one of these then \mkfms\ will try to read the
parameters from a file.
\end{clo}

\begin{clo}{bright}
The brightest absolute magnitude for calculation. The default is --7.0.
\end{clo}

\begin{clo}{faint}
The faintest absolute magnitude for calculation. The default is 17.0.
\end{clo}

\begin{clo}{delta}
The absolute magnitude interval for calculation. The default is 0.5.
\end{clo}

%%----------------------------------------------------------------------------|
\subsection*{Notes}

\mkfms\ has parameters for six filters built in $B$, $V$, $R$, $I$, $J$
and $K$. The band name is case sensitive -- $v$ is not the same as $V$. If the
``filter'' is not one of these then \mkfms\ will assume it it a file name and 
will try to read it to get the parameters. The format of this one line file is
just five, whitespace separated floating point numbers:

\begin{tabbing}
\rule{1cm}{0cm} \= \rule{1cm}{0cm} \= \rule{1cm}{0cm} \= \rule{1cm}{0cm} \= \rule{1cm}{0cm} \kill
$C$ \> $\alpha$ \> $\beta$ \> $\gamma$ \> $M_a$ 
\end{tabbing}

The output is printed on the standard output so redirection may be required.
The format of the output is a standard \lf\ compatible with the other \gmu\
tools and with {\sc bsm} and {\sc egm}.

%%----------------------------------------------------------------------------|
%% MKMLF                                                                      |
%%----------------------------------------------------------------------------|

\newpage
\section{The \mkmlf\ Utility}

\mkmlf\ is another program to make luminosity functions -- this time from
a given mass function. \mkmlf\ assumes a power law mass function of the form
\[
n({\cal M}) = A {\cal M}^{-(1+x)}
\]
where the normalization $A$ and slope $x$ are given on the command line or
by default values. A mass -- luminosity relation must also be specified (or
redirected to standard input).

%%----------------------------------------------------------------------------|
\subsection*{Usage}

\begin{verbatim}
mkmlf x a bright faint del [mlr]
\end{verbatim}

\nin
The order of the command line arguments cannot be changed. They refer to:
\medskip

\begin{clo}{x}
The slope of the mass function in the above equation.
\end{clo}

\begin{clo}{a}
The normalization factor for the mass function. Use a value of 1.0 if you 
know what normalization is required -- the \mknlf\ utility can be used to 
re--normalize the \lf.
\end{clo}

\begin{clo}{bright}
The brightest absolute magnitude for output.
\end{clo}

\begin{clo}{faint}
The faintest absolute magnitude for output.
\end{clo}

\begin{clo}{delta}
The absolute magnitude interval for output.
\end{clo}

\begin{clo}{mlr}
This is the mass--luminosity relation to be used. If it is not given \mkmlf\
will try to read from standard input.
\end{clo}

%%----------------------------------------------------------------------------|
\subsection*{Notes}

The \mlr\ is considered to be undefined for values outside the limits of the
\mlr\ file -- that is, no extrapolation is performed. This means that the
default values for the mass range are set by the \mlr\ file itself, and that
masses beyond this range are not used.

\mkmlf\ reads the \mlr\ in the form of a file. Again a two column, space
separated format is used: the first column is the absolute magnitude; the
second is the mass. The output is in the same format as other \lf s and 
appears on the standard output stream.

%%----------------------------------------------------------------------------|
%% MKNLF                                                                      |
%%----------------------------------------------------------------------------|

\newpage
\section{The \mknlf\ Utility}

This utility is used to normalize luminosity functions. The files must be in 
the format used by {\sc bsm}, {\sc egm} and all the other utilities. 
\mknlf\ uses an extended form of Simpson's Rule to integrate the input \lf. 
It then adjusts this \lf\ so that the integral will equal the normalization 
given on the command line (in stars per cubic parsec).

%%----------------------------------------------------------------------------|
\subsection*{Usage}

\begin{verbatim}
mknlf norm bright faint [lf]
\end{verbatim}

\nin
The order of the command line arguments cannot be changed. They refer to:
\medskip

\begin{clo}{norm}
The normalization in stars per cubic parsec.
\end{clo}

\begin{clo}{bright}
The bright magnitude limit of the normalization. 
\end{clo}

\begin{clo}{faint}
The faint magnitude limit of the normalization. 
\end{clo}

\begin{clo}{lf}
The input \lf. If this is not given then \mknlf\ will try to read from
standard input.
\end{clo}

%%----------------------------------------------------------------------------|
\subsection*{Notes}

\mknlf\ will automatically adjust the values of {\tt bright} and {\tt faint}
to match that of the input \lf\ -- that is no extrapolation is done at either
end. It then normalizes the \lf\ to contain {\tt norm} stars per cubic parsec.
The value of 0.13 (for the solar neighborhood) may be considered an adequate 
default for most purposes.

Part of the normalization process is to integrate the input \lf. The value of
this summation is printed by \mknlf\ but on the standard error stream (which
means that it won't get redirected with the output \lf). This can be useful
for checking \mknlf.

If {\tt bright} and {\tt faint} are the same, \mknlf\ will use the value
of that magnitude bin to normalize the \lf\ -- no integration will be done.

The format for both the input and output \lf s is the same, and matches the
formats given for {\sc bsm} and the other utilities.

%%----------------------------------------------------------------------------|
%% MKSPL                                                                      |
%%----------------------------------------------------------------------------|

\newpage
\section{The \mkspl\ Utility}

\mkspl\ is a tool that fits a cubic spline to the input file and writes
the result to standard out. The fit is performed between the {\tt min} and
{\tt max} as specified on the command line, with an interval of {\tt del}.
If these limits are past the limits of the input the cubic spline is likely
to give strange results. The {\tt -l} option tells \mkspl\ to extend the fit
past the end points by using linear extrapolation. Values less than {\tt min}
are derived using the first and second points; values greater than {\tt max}
use the last and penultimate points. The {\tt -e} options tells \mkspl\ to
replace values past {\tt max} by the last input point, and values past 
{\tt min} with the first -- in other words in {\em extends} the input file.

%%----------------------------------------------------------------------------|
\subsection*{Usage}

\begin{verbatim}
mkspl [-l|-e] min max del [file]
\end{verbatim}

\nin
The order of the command line arguments cannot be changed. They refer to:
\medskip

\begin{clo}{-l}
Extend points past the input function limits using linear extrapolation.
\end{clo}

\begin{clo}{-e}
Extend points past the input function limits using the first or last point
in the input function.
\end{clo}

\begin{clo}{min}
Staring point for the cubic spline fit.
\end{clo}

\begin{clo}{max}
End point of the cubic spline fit.
\end{clo}

\begin{clo}{del}
The interval for the fit.
\end{clo}

%%----------------------------------------------------------------------------|
\subsection*{Notes}

The input file (two column data) must be sorted increasing in the first
column. The end points of the input file do not have to match up with 
{\tt max} and {\tt min}: the {\tt -l} and {\tt -e} options are provided to 
stop the fit going weird in such a case.

Note that \mkspl\ is fairly generic and can be used on {\em any} file of the
right format. It does not produce a comment line on output so that it can be
used in a variety of situations.


%%----------------------------------------------------------------------------|
%% MKXLF                                                                      |
%%----------------------------------------------------------------------------|

\newpage
\section{The \mkxlf\ Utility}

The \mkxlf\ utility produces a \lf\ with a given slope. It is mainly used to
create test data, and is usually used in conjunction with \mknlf. It produces
a \lf\ according to:
\[
N = M^{x} 
\]
between {\tt bright} and {\tt faint} at an interval of {\tt del} (all given
on the command line). The output is a \lf\ that can be used in {\sc bsm}, 
{\sc egm} and all the other utilities.

%%----------------------------------------------------------------------------|
\subsection*{Usage}

\begin{verbatim}
mkxlf x bright faint del
\end{verbatim}

\nin
The order of the command line arguments cannot be changed. They refer to:
\medskip

\begin{clo}{x}
The slope of the luminosity function.
\end{clo}

\begin{clo}{bright}
The brightest magnitude on output.
\end{clo}

\begin{clo}{faint}
The faintest magnitude on output.
\end{clo}

\begin{clo}{del}
The magnitude interval to use for the output \lf.
\end{clo}

%%----------------------------------------------------------------------------|
\subsection*{Notes}

None. It's too simple to have notes!

%%----------------------------------------------------------------------------|
%% EGMCMP                                                                     |
%%----------------------------------------------------------------------------|

\newpage
\section{The \egmcmp\ Utility}

The \egmcmp\ program may be of limited use -- it was written for a specific
purpose by the author, but may be used as the basis for other users to adapt.
It compares an \egm\ model file with a file containing observed data and 
computes the $\chi^2$ statistic.

%%----------------------------------------------------------------------------|
\subsection*{Usage}

\begin{verbatim}
egmcmp [-cqrv#] d|s|t bright faint egm\_file dat\_file [egm\_file dat\_file...]
\end{verbatim}

\nin
The order of the command line arguments cannot be changed. The optional
arguments must be grouped together. Pairs of \egm\ and comparison files may
analysed -- each pair will be read and the results added to the ``running''
$\chi^2$ value. The arguments refer to:
\medskip

\begin{clo}{-c}
Print out the $\chi^2$ statistic. This is the default.
\end{clo}

\begin{clo}{-q}
Print out the $\chi^2$ probability, $Q(\chi^2|\nu)$.
\end{clo}

\begin{clo}{-r}
Print out the reduced $\chi^2$ statistic, $\chi^2/\nu$.
\end{clo}

\begin{clo}{-v}
Turns on verbose mode -- each line of data being used to find $\chi^2$ is
displayed, along with $Q(\chi^2|\nu)$ and $\chi^2$.
\end{clo}

\begin{clo}{-\#}
Here {\tt \#} represents a number -- the default is 1. It sets the method to
be used to determine $\chi^2$. See the source code ({\tt egmcmp.c}) to find
out which method does what.
\end{clo}

\begin{clo}{d}
Uses the disk counts in the comparison.
\end{clo}

\begin{clo}{s}
Uses the spheroid counts in the comparison.
\end{clo}

\begin{clo}{t}
Uses the total of the disk and spheroid counts in the comparison.
\end{clo}

\begin{clo}{bright}
The brightest magnitude bin to include in the comparison.
\end{clo}

\begin{clo}{faint}
The faintest magnitude bin to include in the comparison.
\end{clo}

\begin{clo}{egm\_file}
The \egm\ file containing the model results. Only the number count section is
read, and currently only 2 component models are supported.
\end{clo}

\begin{clo}{dat\_file}
The data file for comparison. 
\end{clo}

%%----------------------------------------------------------------------------|
\subsection*{Notes}


%%----------------------------------------------------------------------------|
%% ADD2EGM                                                                    |
%%----------------------------------------------------------------------------|

\newpage
\section{The \addegm\ Utility}

This does...

%%----------------------------------------------------------------------------|
\subsection*{Usage}

\begin{verbatim}
addegm [-comp] bright faintt < egm\_file
\end{verbatim}

\nin
The order of the command line arguments cannot be changed. They refer to:
\medskip

\begin{clo}{comp}
comp is one of d s t
\end{clo}

\begin{clo}{bright}

\end{clo}

\begin{clo}{faint}

\end{clo}


%%----------------------------------------------------------------------------|
\subsection*{Notes}





%%----------------------------------------------------------------------------|
%% APPENDIX A                                                                 |
%%----------------------------------------------------------------------------|

\newpage
\section*{Appendix A -- Installing GMU}

Installation is fairly straightforward, consisting of unpacking the 
distribution archive and typing ``make''. The latest \gmu\ should be archived
as ``gmu-X.X.tar.gz'' or ``GmuSrcXX.zip'', where the XX is the version number.
The current version is 1.0. If you have any problems contact the author via 
e-mail at {\tt hodder@geop.ubc.ca}. Assuming you have, or can get, one of
these files (available from the author or via FTP) you should do the following:

\begin{enumerate}
\item Ensure you have an ANSI C compiler on your system.  The GNU  compiler
gcc works just fine (\gmu\ was developed with it). If you don't have one 
you're out of luck. Talk to your local UNiX wizard.

\item Make a directory for the source code.
\begin{verbatim}
$ mkdir /home/fred/gmusrc
\end{verbatim}

\item Change to that directory and move the \gmutar\ file into it.
\begin{verbatim}
$ cd gmusrc
$ mv ../gmu-1.2.tar.gz .
\end{verbatim}

\item If you have the tarred, compressed file then uncompress and untar it (!):
\begin{verbatim}
$ zcat gmu-1.2.tar.gz | tar xvf -
\end{verbatim}

\item If you have the zipped file and access to unzip:
\begin{verbatim}
$ unzip GmuSrc12.zip
\end{verbatim}

\item Unpacking the file will create  several files, most of them C code.
Edit the Makefile. Change the definition of BINDIR to where you want the
executables to live.

\item Make the executables:
\begin{verbatim}
$ make all
\end{verbatim}

\item If everything  works O.K  there  should  be no  error messages during
compilation. If not then see below for troubleshooting.

\item Clean up the src directory:
\begin{verbatim}
$ make clean
\end{verbatim}

\item Install the programs:
\begin{verbatim}
$ make install
\end{verbatim}

\item If you have \LaTeX\ installed on your system make the manual:
\begin{verbatim}
$ make manual
\end{verbatim}
\end{enumerate}

%%----------------------------------------------------------------------------|
\subsection*{Troubleshooting}

Most of the problems with the \gmu\ will probably occur during compilation.
\gmu\ uses many ANSI C constructs so it must be compiled with an ANSI C 
compatible complier (such as gcc). If you don't have gcc, or  something 
like it, you can try to get a conversion  program to convert ANSI to K\&R 
(traditional) C.

%%----------------------------------------------------------------------------|
\subsection*{Running GMU at UBC}

If you are at UBC you do not need to do any installation -- the current
version of the \gmu\ live in /home/hodder/bin. Add this to your path if you
want, or copy the executables where you will. 

\end{document}
